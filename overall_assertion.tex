\documentclass[12pt, fleqn]{article}

% for indentation in the first paragraph after a section title
\usepackage{indentfirst}
\usepackage{amssymb}
\usepackage{amsmath}
\usepackage{titlesec}
\usepackage{fancyhdr}
\usepackage{listings} % for code
\usepackage{url} % the urls..
%\usepackage{hyperref}

\DeclareMathOperator*{\SIGMA}{\Sigma}

%\pagestyle{fancy}
%\lhead{The ``new idea''}
%\chead{}
%\rhead{Xuankang Lin}

\begin{document}
%%%%%%%%%%%%%%%%%%%%%%%%%%%%%%%%%%%%%%%%%%%%%%%%%%%%%%%%%%%%%%%%%%%%%%%%%%%%%%%%

Assuming the commands and assertions are stated as following:

\begin{lstlisting}
    { P0: PRE }
      c1
    { P1 }
      c2
    { P2 }
      ..
    { Pn-1 }
      cn
    { Pn: POST }
\end{lstlisting}

%%%%%%%%%%%%%%%%%%%%%%%%%%%%%%%%%%%%%%%%%%%%%%%%%%%%%%%%%%%%%%%%%%%%%%%%%%%%%%%%

\section{Using the ``Overall Assertion''}

\subsection{Goal}

Assuming we have an ``Overall Assertion'' $A$ which is related to all
$P_i$ where $i \in [0, n]$, such that if $A$ is stable w.r.t. some rely,
each $P_i$ will be stable w.r.t. that rely.

The property can be stated as the theorem below:

\begin{equation*}
\begin{aligned}
$Theorem &Representative:$ \forall A \ P \ c,\\
&A = $ ``Overall Assertion'' of all the assertions in code $ \rightarrow\\
&A $ is stable w.r.t. $ Rely(P, c) \rightarrow\\
&\forall $ assertion $ P_i $ in the code, $ P_i $ is stable w.r.t. $ Rely(P, c).
\end{aligned}
\end{equation*}

where $Rely(P, c)$ is a rely such that if the pre-condition $P$ is
sastified, the command $c$ may be executed by some other thread.

\bigskip

Checking stability using such an ``Overall Assertion'' $A$ could cost
less time. For example:

$P_0: \{ x = 0 \}, P_1: \{ x = 1 \}, P_2: \{ x = 2 \} .. P_n: \{ x = n
\}, A: \{ 0 \le x \le n\}$. So in this case, stability check only
needs to check $A$ for once, instead of checking each $P_i$ which is
$n$ times.

%%%%%%%%%%%%%%%%%%%%%%%%%%%%%%%%%%%%%%%%

\subsection{If $A$ is not stable}

But if $A$ is not stable w.r.t. some rely, it may still be the case that
all assertion $P_i$ are stable w.r.t. that rely. This depends on how we
generate $A$.

\subsubsection{Without relaxation}

If $A$ is generated without any relaxation or generalization, then
$A$'s failure of stability should indicate that there is some
assertion which will fail the stability check.

\begin{equation*}
\begin{aligned}
$Theorem &NoRelaxation:$ \forall A \ P \ c,\\
&A = $ ``Overall Assertion'' of all assertions in code without relaxation $ \rightarrow\\
&A $ is not stable w.r.t. $ Rely(P, c) \rightarrow\\
&\exists i, P_i $ is not stable w.r.t. $ Rely(P, c).
\end{aligned}
\end{equation*}

One such example: $P_0: \{ x = 0 \}$, $P_1: \{ x = 10 \}$, $A: \{ x
\in \{0, 10\} \}$. In this trivial case, $A$ does not have any
relaxation or generalization, and the $NoRelaxation$ theorem should
apply to this case.

However, it should be quite hard to generate $A$ for cases that are
not this simple. Also, it could take more time generating such an $A$,
so there is a tradeoff here.

%%%%%%%%%%%%%%%%%%%%

\subsubsection{With relaxation}

If $A$ is generated with some relaxation or generalization, then $A$'s
failure of stability indicates nothing, it could still be the case
that all assertions $P_i$ are stable w.r.t. the rely.

A trivial counterexample for this is stated below:

$P_0: \{ x = 0 \}$, $P_1: \{ True \}$, $P_2: \{ x = 2 \}$. $A: \{ 0
\le x \le 2 \}$. Given a rely $r = Rely(\{ x = 1 \}, x := 100)$, then
$A$ is not stable w.r.t. $r$, but actually each assertion $P_i$ is
stable w.r.t. $r$.

The coarser the generation of $A$ is, the more likely that $A$ fails
the stability check while all $P_i$ are indeed stable. Again, there is
a tradeoff here.

%%%%%%%%%%%%%%%%%%%%

\subsubsection{After failure}

When $A$ fails the stability check, one naive approach is to roughly
divide the assertions into two halves and redo the generation and
checking on each half (similar to binary search).

We can also employ a smarter approach, first find out which subparts
in $A$ make it unstable. For example, if $A$ is in the form of $A_1
\land A_2 \land .. \land A_k \land ..$ and $A_k$ is the subpart that
makes $A$ unstable. Then for all the assertions $P_i$ that can imply
$A_k$, their ``Overall Assertion'' $A'$ will also fail the stability
check.

Then we can eliminate those assertions for now and check other
``benign'' assertions first. The failed original assertions can be
checked one by one.

%%%%%%%%%%%%%%%%%%%%%%%%%%%%%%%%%%%%%%%%%%%%%%%%%%%%%%%%%%%%%%%%%%%%%%%%%%%%%%%%

\section{Generating the ``Overall Assertion''}

The key problem of course is how to generate such an assertion.

\bigskip

% TODO Which one is true:
% 1. Given only one assertion, use our own predicate transformer to generate everything
% 2. All assertions ready (or given one assertion and use wlp/sp to generate others).

If only one assertion is given, we can use wlp/sp to generate all the
assertions. So we can just assume that all assertions are in place.

Let pre-condition $P_0$ be the first assertion (before $c_1$), $P_i$
be the assertion after command $c_i$. $P_n$ is also the
post-condition. Then we define some function $combine$ that combines
all assertions in code into one assertion.

\bigskip

By summerizing in the gc algorithm proof draft, there are at least
$12$ types of assertions:

(aexp are arithmatic expressions)

\begin{equation*}
\begin{aligned}
assertion &::=\\
&| \ \forall var, assertion\\
&| \ \exists var, assertion\\
&| \ assertion \land assertion\\
&| \ assertion \lor assertion\\
&| \ assertion \implies assertion\\
&| \ aexp = aexp\\
&| \ aexp \neq aexp\\
&| \ aexp \le aexp\\
&| \ aexp \ge aexp\\
&| \ aexp \mapsto aexp\\
&| \ \{ .. \} \subseteq \{ .. \}\\
&| \ aexp \in \{ .. \}\\
\end{aligned}
\end{equation*}

One implementation (draft): $A$ is in the form of $A_1 \land A_2 \land
.. \land A_k ..$ where each $A_i$ is a combined assertion w.r.t. one
variable.

When a new assertion $P_i$ is to be combined into an ``Overall
Assertion'' $A$, if there is already some $A_k$ that handles with
$P_i$'s variables, generalize $A_k$ to make it also covers
the case of $P_i$. Otherwise, $A = A \land P_i$.

Example: $P_0: \{ x = 10 \}$, $P_1: \{ x = 11 \land y = 100\}$, $P_2:
\{ y = 101 \land z = 1000 \}$. Then $A$ after each step is:

\begin{itemize}
\item Before combining $P_0$ (initially): $A: True$
\item After combining $P_0$: $A: x = 10$
\item After combining $P_1$: $A: 10 \le x \le 11 \land y = 100$
\item After combining $P_2$ (finally): $A: 10 \le x \le 11 \land 100 \le y \le 101 \land z = 1000 $
\end{itemize}

\textbf{TODO} The difficult cases are those disjunction combinations
and $\forall / \exists$ quantifiers, such as combining $x = 100$ and
$x = 10 \lor y = 20$. We need to set rules for combining these
different shapes of assertions and specify a clear algorithm.

\bigskip

After $combine$, we get a ``Overall Assertion'' $A$. $A$ should have
the following property: $\forall$ assertion $P_i, P_i \implies
A|_{P_i}$.  Here $A|_{P_i}$ represents those subparts in $A$ that are
corresponding to $P_i$. e.g. $A: \{x > 10 \land y < 100\}, P_i: \{x =
11\} \therefore A|_{P_i} = \{x > 10\}$.

Intuitively, the ``Overall Assertion'' $A$ is a combination of
assertions concerning all variables as well as a generalization in
terms of one single variable.

%%%%%%%%%%%%%%%%%%%%%%%%%%%%%%%%%%%%%%%%%%%%%%%%%%%%%%%%%%%%%%%%%%%%%%%%%%%%%%%%

\section{(Perhaps) Related papers}

We searched for papers that may have done similar things like
combining assertions and use the result to represent its original
assertions.

Keywords used so far: predicate tranformers, composition, synthesis,
etc.

But most papers we read so far are not talking things we are
interested in here..

\subsection{\textsl{win} and \textsl{sin}: Predicate Transformers for Concurrency}

\url{http://dl.acm.org/citation.cfm?id=78970}

\bigskip

In this paper, \textsl{win} and \textsl{sin} predicate transformers
are defined to extend the Owicki-Gries method to allow reasoning about
programs without specifying atomic actions.

Owicki-Gries methods requires the operation to be atomic in order to
be reasoned about. Lamport loosened this requirement in this paper so
that non-atomic operations can also use Owick-Gries method for
reasoning.

Lamport defines everything from scratch. Two invariants $I$ and $I'$
are defined to describe the state before and after some operation,
respectively. The operation here may not be atomic, in other words it
may be a set of atomic operations. Then Lamport uses $I$ and $I'$ for
Owicki-Gries method just as they are pre-condition and post-condition.

So this paper may not be relevant to our idea.

%%%%%%%%%%%%%%%%%%%%%%%%%%%%%%%%%%%%%%%%

\subsection{Weakest Precondition Synthesis for Compiler Optimizations}

\url{http://link.springer.com/chapter/10.1007/978-3-642-54013-4_12}

\bigskip

This paper presents an algorithm which can generate the weakest
(liberal) pre-conditions for some compiler optimizations w.r.t. some
template programs such that when the generated weakest preconditions
are satisfied, the category of optimizations on these programs won't
introduce new bugs.

This paper may not be relevant to our idea.

%%%%%%%%%%%%%%%%%%%%%%%%%%%%%%%%%%%%%%%%

\subsection{Reasoning about Program Composition}

\url{https://www.cise.ufl.edu/tr/DOC/REP-1996-233.pdf}

\bigskip

The word ``composition'' in the title actually means ``parallel
composition'', $||$ is the composition operator. So the
``composition'' in this paper is like taking the programs in different
threads altogether into consideration.

They also define some ``guarantee'' operator using $||$ and also a
predicate transformer $wg$ (weakest guarantee).

$(P$ guarantees $Q).F$ means $\forall H, P$ holds in $F || H \implies
Q$ holds in $F || H$. $wg.F.Q$ is the weakest P such that $(P$
guarantees $Q).F$.

Here $F$ and $H$ are programs, $P$ and $Q$ are the predicates on
program $F$. (They also have predicates on states). There is no
definition for ``program'', at least not in this paper. I assume this
``program'' is equivalent to one command.

So this paper is like providing a new theory for reasoning concurrent
programs. But it may be irrelevant to our idea.

%%%%%%%%%%%%%%%%%%%%%%%%%%%%%%%%%%%%%%%%

\subsection{Reasoning about Composition Using Property Transformers and Their Conjugates}

\url{http://link.springer.com/chapter/10.1007/3-540-44929-9_40}

\bigskip

This paper is also from K. Mani Chandy as the previous one (Reasoning
about Program Composition). Comparing to the previous paper, this one
is more general, ``composition'' no longer solely refers to ``parallel
composition''. Instead, it's ``component composition''.

This paper defines 6 predicate transformers which may be interesting:

\begin{itemize}
\item $WE.X$: What must be proved on a component to ensure that any
  system that contains that component satisfies X.
\item $SE.X$: What can be deduced of any system that contains at least
  one component that satisfies X.
\item $SU.X$: What can be deduced of any system that contains only
  components that satisfy X.
\item $WE^*.X$: What can be deduced on all components of any system
  that satisfies X.
\item $SE^*.X$: What must be proved on a system to ensure that all
  components satisfy X.
\item $SU^*.X$: What must be proved on a system to ensure that at
  least one component satisfies X.
\end{itemize}

However, this paper is more like just presenting a model and talks
about its methodology, it is irrelevant to the things I am interested
in -- rules for assertion combination.

%%%%%%%%%%%%%%%%%%%%%%%%%%%%%%%%%%%%%%%%

\subsection{Explicit Stabilisation for Modular Rely-Guarantee Reasoning}

\url{http://link.springer.com/chapter/10.1007\%2F978-3-642-11957-6_32}

\bigskip

$\lceil P \rceil_R$ and $\lfloor P \rfloor_R$ are defined in the paper
to represent the strongest assertion that is weaker than $P$ as well
as stable under $R$, and the weakest assertion that is stronger than
$P$ as well as stable under $R$, respectively.

In this way, the rely $R$ can be regarded as a parameter. So it may be
possible to generate assertions with $R$ unspecified and just check
the generated pre-condition and post-condition with actual relies.

If this is possible, they are to some extent similar to our idea in
this summary. I am still working on applying that idea to the gc
algorithm proof draft to see if that actually works.

%%%%%%%%%%%%%%%%%%%%%%%%%%%%%%%%%%%%%%%%%%%%%%%%%%%%%%%%%%%%%%%%%%%%%%%%%%%%%%%%
\end{document}

